\documentclass{supervision}
\usepackage{course}

\Supervision{2}

\begin{document}

    \begin{questions}
        \section*{Topic 02 - Architecture and Internet}

        \SetQuestionNumber{4}
        \question
        \emph{Standards - so many to choose from and Internet Philosophy}
        \begin{parts}
            \part
            The Internet’s standards body, the IETF, has a philosophy which was summarised by David Clark, one of the Internet’s pioneers, as follows
            \begin{quote}
            “We reject kings, presidents and voting. We believe in rough consensus and running code.”
            \end{quote}
            This suggests an approach which is open, dynamic and led by implementation. By contrast, other standards bodies such as the ITU are closed, slow-moving and led by specification. Using examples, discuss ways in which the IETF’s approach has enabled innovation in the Internet, and ways in which it has caused problems.
            \part
            Prior to the Internet, wide-area networks were joined together at level of application protocols, using gateways. Explain, as fully as you can, why this approach limited application development.
            \part
            Explain how the design of the Internet protocol, i.e. IP, addressed this problem of application development. You should explain how the term “hourglass model” describes IP’s approach to network layering.
            \part
            The design of IP makes explicit provision for fragmentation, i.e. the ability to split an individual packet into pieces during its journey across the network. By considering the hourglass model, suggest why this feature is essential.

        \end{parts}

        \question
        \begin{parts}
            \part
            What is the difference between an architectural principle and an architectural design (choice)? What other examples can you think of?
            \part
            A NAT stores state about the flows (connections) that pass through it.
            \begin{subparts}
                \subpart
                Why does a NAT break the \emph{end-to-end} principle?
                \subpart
                Why might the NAT violation of \emph{end-to-end} principles not
                actually matter?
            \end{subparts}
        \end{parts}

    \end{questions}

\end{document}
