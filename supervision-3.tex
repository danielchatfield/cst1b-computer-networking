\documentclass{supervision}
\usepackage{course}
\Supervision{3}

\begin{document}
  \begin{questions}
    %%%%%%%%%%%%%%%%%%%%%%%%%%%%%%%%%%%%%%%%%%
    \section*{Topic 03 - Data-Link (Physical)}
    %%%%%%%%%%%%%%%%%%%%%%%%%%%%%%%%%%%%%%%%%%
    \SetQuestionNumber{7}
    \question \textit{Shared media multiplexing in local area networks}
    \begin{parts}
      \part Define the term \textit{shared media network}

      \part Explain how Ethernet performs \textit{carrier sense},
      \textit{collision detection}, how it aims to minimise the
      probability of collision on retransmission and how this is adapted
      to handle varying load.

      \part Explain why token ting does not share media at the physical
      level, but is still analyzed as a shared media system.

      \part What is the role of a token monitor in token ring? Why does the
      monitor \textit{not} prevent failure when one of the nodes in the
      ring suffers a hardware failure? Suggest how a token ring system
      might be designed to handle failure of a computer attached to the
      ring.

      \part Explain the meaning of \textit{destination delete} and
      \textit{source delete} as used in ring-based networks.

      \part Explain the difference between conventional token rings and
      \textit{slotted rings}. What are the advantages of a slotted ring?
    \end{parts}

    \question \textit{Multiplexing redux}
    \begin{parts}
      \part Several real-time video streams are to share the same
      lower-layer channel.
      \begin{subparts}
        \subpart Give one example of a lower-layer channel in which the
        flows might be scheduled, and one in which scheduling is not
        possible.

        \subpart A lecturer remarks that “centralised multiplexing”
        offers potential gains in efficiency over non-centralised
        multiplexing. Give two reasons why this can improve efficiency.
        What, in general terms, is the “centralised” facility necessary
        for these gains to be possible?

        \subpart Using an example, describe why specifying a scheduling
        policy in terms of priority may cause problems, even where it
        is safe to use priority within the scheduling mechanism.
        [Hint: consider CPU scheduling in an operating system.]
      \end{subparts}

      \part Code-division multiple access (CDMA) is a code-division
      multiplexing system, used for mobile telephony.
      \begin{subparts}
        \subpart What is a code? What property of codes causes them to be
        “nearly orthogonal” to each other?

        \subpart Two transmitters, A and B, both want to transmit a
        four-bit message at the same time using CDMA. Transmitter A has
        code 10010111 and message 1001. Transmitter B has code 00111101
        and message 0011. Write down the bit sequences transmitted by A
        and B. Write down the bit sequence seen by a receiver, stating
        any assumption you make. Show that the original messages of
        both A and B may be recovered. [Each bit is transmitted as the
        exclusive OR of the code sequence with the bit value.]
      \end{subparts}
    \end{parts}

    \question \textit{Coding, digitisation, error detection and error
    correction}
    \begin{parts}
      \part Give, with examples, three advantages of digitising audio, and
      three corresponding disadvantages. (Compare with storing and
      processing it exclusively on analogue media and equipment.)

      \part Explain quantisation and sampling of analogue signals, and the
      distinction between these. State an upper bound for the
      signal-to-noise ratio of a signal quantised at $b$ bits resolution,
      assuming the analogue original to be noiseless and the quantisation
      process completely accurate.

      \part Outline encode and decode procedures for a simple ($m$,$k$)
      block code. Show that the minimum distance of any simple checksum
      code is always $2$.
    \end{parts}

    \question \textit{CRCs}
    \begin{parts}
      \part Explain, giving an example, how to write a binary message (i.e.
      a sequence of binary digits) as a polynomial.

      \part Outline send and receive procedures for CRC-based message
      coding and error detection. What information must be agreed in
      advance by the sender and reciever?

      \part Draw a shift register which will compute the remainder on
      division of an input polynomial by the CRC-8 polynomial
      $x^8 + x^2 + x^1 +1$.

    \end{parts}
  \end{questions}
\end{document}
